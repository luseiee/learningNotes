\documentclass{article}
% generated by Madoko, version 1.0.3
%mdk-data-line={1}


\usepackage[heading-base={2},section-num={False},bib-label={True}]{madoko2}


\begin{document}



%mdk-data-line={5}
\mdxtitleblockstart{}
%mdk-data-line={5}
\mdxtitle{\mdline{5}Classic Shell Scripting Notes}%mdk
\mdxauthorstart{}
%mdk-data-line={10}
\mdxauthorname{\mdline{10}Lu, Phil}%mdk
\mdxauthorend\mdtitleauthorrunning{}{}\mdxtitleblockend%mdk

%mdk-data-line={7}
\section{\mdline{7}1.\hspace*{0.5em}\mdline{7}背景知识}\label{section}%mdk%mdk

%mdk-data-line={9}
\begin{itemize}[noitemsep,topsep=\mdcompacttopsep]%mdk

%mdk-data-line={9}
\item\mdline{9}POSIX

%mdk-data-line={10}
\begin{quote}%mdk
\mdline{10}POSIX,Portable Operating System Interface。\mdline{10}\mdbr
\mdline{11}是UNIX系统的一个设计标准,很多类UNIX系统也在支持兼容这个标准,如Linux。\mdline{11}\mdbr
\mdline{12}遵循这个标准的好处是软件可以跨平台。所以windows也支持就很容易理解了,那么多优秀的开源软件,支持了这个这些软件就可能有windows版本,就可以完善丰富windows下的软件。
%mdk
\end{quote}%mdk%mdk
%mdk
\end{itemize}%mdk

%mdk-data-line={14}
\section{\mdline{14}2.\hspace*{0.5em}\mdline{14}入门}\label{section}%mdk%mdk

%mdk-data-line={15}
\subsection{\mdline{15}2.1.\hspace*{0.5em}\mdline{15}\#\mdline{15}!}\label{section}%mdk%mdk
\begin{mdpre}%mdk
\noindent\preindent{2}cat~\textgreater{}~nursers\\
\preindent{2}\#!~/bin/bash~-\\
\preindent{2}\\
\preindent{2}who~\textbar{}~wc~-l%mdk
\end{mdpre}\noindent\mdline{22}之后每条命令都会用bash运行
所以,如果是py文件可以在第一行加上\mdline{23}\#\mdline{23}! /bin/python, 这样就可以使用./来运行它

%mdk-data-line={24}
\subsection{\mdline{24}2.2.\hspace*{0.5em}\mdline{24}printf}\label{sec-printf}%mdk%mdk

%mdk-data-line={25}
\begin{itemize}[noitemsep,topsep=\mdcompacttopsep]%mdk

%mdk-data-line={25}
\item\mdline{25}printf 命令
用法与C语言非常接近%mdk
%mdk
\end{itemize}%mdk

%mdk-data-line={28}
\subsection{\mdline{28}2.3.\hspace*{0.5em}\mdline{28}重定向与管道}\label{section}%mdk%mdk

%mdk-data-line={29}
\subsubsection{\mdline{29}2.3.1.\hspace*{0.5em}\mdline{29}\textless{}\mdline{29}~\mdline{29}改变标准输入}\label{sec--}%mdk%mdk

%mdk-data-line={30}
\begin{quote}%mdk

%mdk-data-line={30}
\noindent\mdline{30}tr命令\mdline{30}\mdbr
\mdline{31}translate charactors\mdline{31} \mdline{31}%mdk
\begin{mdpre}%mdk
\noindent\preindent{2}tr~-d~'0-9'%mdk
\end{mdpre}\noindent\mdline{35}可以将标准输入中的数字全都删除
\begin{mdpre}%mdk
\noindent\preindent{2}tr~-d~'0-9'~\textless{}~mytext.txt%mdk
\end{mdpre}\noindent\mdline{39}这就改变了标准输入而是将txt文件中的内容进行修改
%mdk
\end{quote}%mdk

%mdk-data-line={41}
\subsubsection{\mdline{41}2.3.2.\hspace*{0.5em}\mdline{41}\textgreater{}\mdline{41}~\mdline{41}改变标准输出}\label{sec--}%mdk%mdk

%mdk-data-line={42}
\subsubsection{\mdline{42}2.3.3.\hspace*{0.5em}\mdline{42}\textgreater{}\mdline{42}\textgreater{}\mdline{42}~\mdline{42}输出到文件时不覆盖文件而是附加}\label{sec--}%mdk%mdk

%mdk-data-line={43}
\subsubsection{\mdline{43}2.3.4.\hspace*{0.5em}\mdline{43}\textbar{}\mdline{43}~\mdline{43}建立管道}\label{sec--}%mdk%mdk
\begin{mdpre}%mdk
\noindent\preindent{2}program1~\textbar{}~program2%mdk
\end{mdpre}\noindent\mdline{47}将program1的标准输出修改为program2的标准输入
\begin{mdpre}%mdk
\noindent\preindent{2}tr~-d~'\textbackslash{}r'~\textless{}~mytext.txt~\textbar{}~sort~\textgreater{}~mytext2.txt%mdk
\end{mdpre}\noindent\mdline{51}过程:

%mdk-data-line={53}
\begin{enumerate}[noitemsep,topsep=\mdcompacttopsep]%mdk

%mdk-data-line={53}
\item\mdline{53}tr的标准输入改成mytext.txt\mdline{53} \mdline{53}%mdk

%mdk-data-line={54}
\item\mdline{54}tr的标准输出变为sort的标准输入\mdline{54} \mdline{54}%mdk

%mdk-data-line={55}
\item\mdline{55}sort的标准输出重定向到mytext2.txt\mdline{55} \mdline{55}%mdk
%mdk
\end{enumerate}%mdk

%mdk-data-line={57}
\subsubsection{\mdline{57}2.3.5.\hspace*{0.5em}\mdline{57}/dev/null和/dev/tty}\label{sec-devnulldevtty}%mdk%mdk

%mdk-data-line={58}
\noindent\mdline{58}/dev/null是一个垃圾桶,所有输出到这里的数据都会被扔掉\mdline{58} \mdline{58}%mdk
\begin{mdpre}%mdk
\noindent\preindent{4}echo~12321~\textgreater{}~/dev/null%mdk
\end{mdpre}\noindent\mdline{62}/dev/tty将重定向一个终端,键盘输入maybe,所以这个是用来作为输入的
\begin{mdpre}%mdk
\noindent\preindent{2}read~password~\textless{}~/dev/tty%mdk
\end{mdpre}\noindent\mdline{66}将会强制从/dev/tty中读取数据,一般情况下不写问题也不大貌似

%mdk-data-line={67}
\begin{quote}%mdk

%mdk-data-line={67}
\noindent\mdline{67}stty\mdline{67} \mdline{67}-echo\mdline{67}\mdbr
\mdline{68}可以将输入不显示在屏幕上\mdline{68}\mdbr
\mdline{69}stty echo\mdline{69}\mdbr
\mdline{70}重新显示\mdline{70} \mdline{70}%mdk
%mdk
\end{quote}%mdk

%mdk-data-line={72}
\subsection{\mdline{72}2.4.\hspace*{0.5em}\mdline{72}Shell脚本的参数}\label{sec-shell}%mdk%mdk

%mdk-data-line={73}
\noindent\mdline{73}\$\mdline{73}1代表第一个参数
\mdline{74}\$\mdline{74}2代表第二个参数%mdk
\begin{mdpre}%mdk
\noindent\preindent{2}cat~\textgreater{}~finduser\\
\preindent{2}\#!~/bin/sh\\
\preindent{2}\\
\preindent{2}who~\textbar{}~grep~\$1\\
\preindent{2}\textasciicircum{}D%mdk
\end{mdpre}\noindent\mdline{82}使用的时候就可以./finduser lxc

%mdk-data-line={84}
\subsection{\mdline{84}2.5.\hspace*{0.5em}\mdline{84}简单的执行跟踪}\label{section}%mdk%mdk

%mdk-data-line={85}
\noindent\mdline{85}set\mdline{85} \mdline{85}-x可以设置是否跟踪命令,跟踪命令是指每条命令执行时候在前面价一个\mdline{85} \mdline{85}+ 并显示出来%mdk

%mdk-data-line={87}
\subsection{\mdline{87}2.6.\hspace*{0.5em}\mdline{87}.profile .bashrc区别}\label{sec-profile-bashrc}%mdk%mdk

%mdk-data-line={88}
\noindent\mdline{88}.profile是每次登陆时运行\mdline{88}\mdbr
\mdline{89}.bashrc是每次运行bash时运行%mdk

%mdk-data-line={91}
\begin{mdbmargintb}{4em}{}%mdk
\begin{mdflushright}%mdk
{\tiny\mdline{92}Created with~\href{https://www.madoko.net}{Madoko.net}.}%mdk
\end{mdflushright}%mdk
\end{mdbmargintb}%mdk%mdk


\end{document}
