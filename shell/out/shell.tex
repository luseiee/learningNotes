\documentclass{article}
% generated by Madoko, version 1.0.3
%mdk-data-line={1}


\usepackage[heading-base={2},section-num={False},bib-label={True}]{madoko2}
\usepackage{ctex}


\begin{document}



%mdk-data-line={6}
\mdxtitleblockstart{}
%mdk-data-line={6}
\mdxtitle{\mdline{6}Classic Shell Scripting Notes}%mdk
\mdxauthorstart{}
%mdk-data-line={11}
\mdxauthorname{\mdline{11}Lu, Phil}%mdk
\mdxauthorend\mdtitleauthorrunning{}{}\mdxtitleblockend%mdk

%mdk-data-line={8}
\section{\mdline{8}1.\hspace*{0.5em}\mdline{8}背景知识}\label{section}%mdk%mdk

%mdk-data-line={9}
\begin{itemize}[noitemsep,topsep=\mdcompacttopsep]%mdk

%mdk-data-line={9}
\item\mdline{9}POSIX

%mdk-data-line={10}
\begin{quote}%mdk
\mdline{10}POSIX,Portable Operating System Interface。\mdline{10}\mdbr
\mdline{11}是UNIX系统的一个设计标准,很多类UNIX系统也在支持兼容这个标准,如Linux。\mdline{11}\mdbr
\mdline{12}遵循这个标准的好处是软件可以跨平台。所以windows也支持就很容易理解了,那么多优秀的开源软件,支持了这个这些软件就可能有windows版本,就可以完善丰富windows下的软件。
%mdk
\end{quote}%mdk%mdk
%mdk
\end{itemize}%mdk

%mdk-data-line={14}
\section{\mdline{14}2.\hspace*{0.5em}\mdline{14}入门}\label{section}%mdk%mdk

%mdk-data-line={15}
\subsection{\mdline{15}2.1.\hspace*{0.5em}\mdline{15}\#\mdline{15}!}\label{section}%mdk%mdk
\begin{mdpre}%mdk
\noindent\preindent{2}cat~\textgreater{}~nursers\\
\preindent{2}\#!~/bin/bash~-\\
\preindent{2}\\
\preindent{2}who~\textbar{}~wc~-l%mdk
\end{mdpre}\noindent\mdline{23}\mdbr
\mdline{24}之后每条命令都会用bash运行
所以,如果是py文件可以在第一行加上\mdline{25}\#\mdline{25}! /bin/python, 这样就可以使用./来运行它

%mdk-data-line={25}
\subsection{\mdline{25}2.2.\hspace*{0.5em}\mdline{25}printf}\label{sec-printf}%mdk%mdk

%mdk-data-line={26}
\begin{itemize}[noitemsep,topsep=\mdcompacttopsep]%mdk

%mdk-data-line={26}
\item\mdline{26}printf 命令
用法与C语言非常接近%mdk
%mdk
\end{itemize}%mdk

%mdk-data-line={29}
\subsection{\mdline{29}2.3.\hspace*{0.5em}\mdline{29}重定向与管道}\label{section}%mdk%mdk

%mdk-data-line={30}
\subsubsection{\mdline{30}2.3.1.\hspace*{0.5em}\mdline{30}\textless{}\mdline{30}~\mdline{30}改变标准输入}\label{sec--}%mdk%mdk

%mdk-data-line={31}
\begin{quote}%mdk

%mdk-data-line={31}
\noindent\mdline{31}tr命令\mdline{31}\mdbr
\mdline{32}translate charactors\mdline{32} \mdline{32}%mdk
\begin{mdpre}%mdk
\noindent\preindent{2}tr~-d~'0-9'%mdk
\end{mdpre}\noindent\mdline{36}可以将标准输入中的数字全都删除
\begin{mdpre}%mdk
\noindent\preindent{2}tr~-d~'0-9'~\textless{}~mytext.txt%mdk
\end{mdpre}\noindent\mdline{40}这就改变了标准输入而是将txt文件中的内容进行修改
%mdk
\end{quote}%mdk

%mdk-data-line={42}
\subsubsection{\mdline{42}2.3.2.\hspace*{0.5em}\mdline{42}\textgreater{}\mdline{42}~\mdline{42}改变标准输出}\label{sec--}%mdk%mdk

%mdk-data-line={43}
\subsubsection{\mdline{43}2.3.3.\hspace*{0.5em}\mdline{43}\textgreater{}\mdline{43}\textgreater{}\mdline{43}~\mdline{43}输出到文件时不覆盖文件而是附加}\label{sec--}%mdk%mdk

%mdk-data-line={44}
\subsubsection{\mdline{44}2.3.4.\hspace*{0.5em}\mdline{44}\textbar{}\mdline{44}~\mdline{44}建立管道}\label{sec--}%mdk%mdk
\begin{mdpre}%mdk
\noindent\preindent{2}program1~\textbar{}~program2%mdk
\end{mdpre}\noindent\mdline{48}将program1的标准输出修改为program2的标准输入
\begin{mdpre}%mdk
\noindent\preindent{2}tr~-d~'\textbackslash{}r'~\textless{}~mytext.txt~\textbar{}~sort~\textgreater{}~mytext2.txt%mdk
\end{mdpre}\noindent\mdline{52}过程:

%mdk-data-line={54}
\begin{enumerate}[noitemsep,topsep=\mdcompacttopsep]%mdk

%mdk-data-line={54}
\item\mdline{54}tr的标准输入改成mytext.txt\mdline{54} \mdline{54}%mdk

%mdk-data-line={55}
\item\mdline{55}tr的标准输出变为sort的标准输入\mdline{55} \mdline{55}%mdk

%mdk-data-line={56}
\item\mdline{56}sort的标准输出重定向到mytext2.txt\mdline{56} \mdline{56}%mdk
%mdk
\end{enumerate}%mdk

%mdk-data-line={58}
\subsubsection{\mdline{58}2.3.5.\hspace*{0.5em}\mdline{58}/dev/null和/dev/tty}\label{sec-devnulldevtty}%mdk%mdk

%mdk-data-line={59}
\noindent\mdline{59}/dev/null是一个垃圾桶,所有输出到这里的数据都会被扔掉\mdline{59} \mdline{59}%mdk
\begin{mdpre}%mdk
\noindent\preindent{4}echo~12321~\textgreater{}~/dev/null%mdk
\end{mdpre}\noindent\mdline{63}/dev/tty将重定向一个终端,键盘输入maybe,所以这个是用来作为输入的
\begin{mdpre}%mdk
\noindent\preindent{2}read~password~\textless{}~/dev/tty%mdk
\end{mdpre}\noindent\mdline{67}将会强制从/dev/tty中读取数据,一般情况下不写问题也不大貌似

%mdk-data-line={68}
\begin{quote}%mdk

%mdk-data-line={68}
\noindent\mdline{68}stty\mdline{68} \mdline{68}-echo\mdline{68}\mdbr
\mdline{69}可以将输入不显示在屏幕上\mdline{69}\mdbr
\mdline{70}stty echo\mdline{70}\mdbr
\mdline{71}重新显示\mdline{71} \mdline{71}%mdk
%mdk
\end{quote}%mdk

%mdk-data-line={73}
\subsection{\mdline{73}2.4.\hspace*{0.5em}\mdline{73}Shell脚本的参数}\label{sec-shell}%mdk%mdk

%mdk-data-line={74}
\noindent\mdline{74}\$\mdline{74}1代表第一个参数
\mdline{75}\$\mdline{75}2代表第二个参数%mdk
\begin{mdpre}%mdk
\noindent\preindent{2}cat~\textgreater{}~finduser\\
\preindent{2}\#!~/bin/sh\\
\preindent{2}\\
\preindent{2}who~\textbar{}~grep~\$1\\
\preindent{2}\textasciicircum{}D%mdk
\end{mdpre}\noindent\mdline{83}使用的时候就可以./finduser lxc

%mdk-data-line={85}
\subsection{\mdline{85}2.5.\hspace*{0.5em}\mdline{85}简单的执行跟踪}\label{section}%mdk%mdk

%mdk-data-line={86}
\noindent\mdline{86}set\mdline{86} \mdline{86}-x可以设置是否跟踪命令,跟踪命令是指每条命令执行时候在前面价一个\mdline{86} \mdline{86}+ 并显示出来%mdk

%mdk-data-line={88}
\subsection{\mdline{88}2.6.\hspace*{0.5em}\mdline{88}.profile .bashrc区别}\label{sec-profile-bashrc}%mdk%mdk

%mdk-data-line={89}
\noindent\mdline{89}.profile是每次登陆时运行\mdline{89}\mdbr
\mdline{90}.bashrc是每次运行bash时运行%mdk

%mdk-data-line={92}
\begin{mdbmargintb}{4em}{}%mdk
\begin{mdflushright}%mdk
{\tiny\mdline{93}Created with~\href{https://www.madoko.net}{Madoko.net}.}%mdk
\end{mdflushright}%mdk
\end{mdbmargintb}%mdk%mdk


\end{document}
