\documentclass{article}
% generated by Madoko, version 1.0.3
%mdk-data-line={1}


\usepackage[heading-base={2},section-num={False},bib-label={True}]{madoko2}


\begin{document}



\mdxtitleblockstart{}
\mdxtitle{Classic Shell Scripting Notes}%mdk
\mdxauthorstart{}
\mdxauthorname{Lu, Phil}%mdk
\mdxauthorend\mdtitleauthorrunning{}{}\mdxtitleblockend%mdk

\section{1.\hspace*{0.5em}背景知识}\label{section}%mdk%mdk

\begin{itemize}[noitemsep,topsep=\mdcompacttopsep]%mdk

\item POSIX

\begin{quote}%mdk
POSIX,Portable Operating System Interface。\mdbr
是UNIX系统的一个设计标准,很多类UNIX系统也在支持兼容这个标准,如Linux。\mdbr
遵循这个标准的好处是软件可以跨平台。所以windows也支持就很容易理解了,那么多优秀的开源软件,支持了这个这些软件就可能有windows版本,就可以完善丰富windows下的软件。
%mdk
\end{quote}%mdk%mdk
%mdk
\end{itemize}%mdk

\section{2.\hspace*{0.5em}入门}\label{section}%mdk%mdk

\subsection{2.1.\hspace*{0.5em}\#!}\label{section}%mdk%mdk
\begin{mdpre}%mdk
\noindent\preindent{2}cat~\textgreater{}~nursers\\
\preindent{2}\#!~/bin/bash~-\\
\preindent{2}\\
\preindent{2}who~\textbar{}~wc~-l%mdk
\end{mdpre}\noindent之后每条命令都会用bash运行
所以,如果是py文件可以在第一行加上\#! /bin/python, 这样就可以使用./来运行它

\subsection{2.2.\hspace*{0.5em}printf}\label{sec-printf}%mdk%mdk

\begin{itemize}[noitemsep,topsep=\mdcompacttopsep]%mdk

\item printf 命令
用法与C语言非常接近%mdk
%mdk
\end{itemize}%mdk

\subsection{2.3.\hspace*{0.5em}重定向与管道}\label{section}%mdk%mdk

\subsubsection{2.3.1.\hspace*{0.5em}\textless{}~改变标准输入}\label{sec--}%mdk%mdk

\begin{quote}%mdk

\noindent tr命令\mdbr
translate charactors%mdk
\begin{mdpre}%mdk
\noindent\preindent{2}tr~-d~'0-9'%mdk
\end{mdpre}\noindent可以将标准输入中的数字全都删除
\begin{mdpre}%mdk
\noindent\preindent{2}tr~-d~'0-9'~\textless{}~mytext.txt%mdk
\end{mdpre}\noindent这就改变了标准输入而是将txt文件中的内容进行修改
%mdk
\end{quote}%mdk

\subsubsection{2.3.2.\hspace*{0.5em}\textgreater{}~改变标准输出}\label{sec--}%mdk%mdk

\subsubsection{2.3.3.\hspace*{0.5em}\textgreater{}\textgreater{}~输出到文件时不覆盖文件而是附加}\label{sec--}%mdk%mdk

\subsubsection{2.3.4.\hspace*{0.5em}\textbar{}~建立管道}\label{sec--}%mdk%mdk
\begin{mdpre}%mdk
\noindent\preindent{2}program1~\textbar{}~program2%mdk
\end{mdpre}\noindent将program1的标准输出修改为program2的标准输入
\begin{mdpre}%mdk
\noindent\preindent{2}tr~-d~'\textbackslash{}r'~\textless{}~mytext.txt~\textbar{}~sort~\textgreater{}~mytext2.txt%mdk
\end{mdpre}\noindent过程:

\begin{enumerate}[noitemsep,topsep=\mdcompacttopsep]%mdk

\item tr的标准输入改成mytext.txt%mdk

\item tr的标准输出变为sort的标准输入%mdk

\item sort的标准输出重定向到mytext2.txt%mdk
%mdk
\end{enumerate}%mdk

\subsubsection{2.3.5.\hspace*{0.5em}/dev/null和/dev/tty}\label{sec-devnulldevtty}%mdk%mdk

\noindent/dev/null是一个垃圾桶,所有输出到这里的数据都会被扔掉%mdk
\begin{mdpre}%mdk
\noindent\preindent{4}echo~12321~\textgreater{}~/dev/null%mdk
\end{mdpre}\noindent/dev/tty将重定向一个终端,键盘输入maybe,所以这个是用来作为输入的
\begin{mdpre}%mdk
\noindent\preindent{2}read~password~\textless{}~/dev/tty%mdk
\end{mdpre}\noindent将会强制从/dev/tty中读取数据,一般情况下不写问题也不大貌似

\begin{quote}%mdk

\noindent stty -echo\mdbr
可以将输入不显示在屏幕上\mdbr
stty echo\mdbr
重新显示%mdk
%mdk
\end{quote}%mdk

\subsection{2.4.\hspace*{0.5em}Shell脚本的参数}\label{sec-shell}%mdk%mdk

\noindent\$1代表第一个参数
\$2代表第二个参数%mdk
\begin{mdpre}%mdk
\noindent\preindent{2}cat~\textgreater{}~finduser\\
\preindent{2}\#!~/bin/sh\\
\preindent{2}\\
\preindent{2}who~\textbar{}~grep~\$1\\
\preindent{2}\textasciicircum{}D%mdk
\end{mdpre}\noindent使用的时候就可以./finduser lxc

\subsection{2.5.\hspace*{0.5em}简单的执行跟踪}\label{section}%mdk%mdk

\noindent set -x可以设置是否跟踪命令,跟踪命令是指每条命令执行时候在前面价一个 + 并显示出来%mdk

\subsection{2.6.\hspace*{0.5em}.profile .bashrc区别}\label{sec-profile-bashrc}%mdk%mdk

\noindent.profile是每次登陆时运行\mdbr
.bashrc是每次运行bash时运行%mdk

\begin{mdbmargintb}{4em}{}%mdk
\begin{mdflushright}%mdk
{\tiny Created with~\href{https://www.madoko.net}{Madoko.net}.}%mdk
\end{mdflushright}%mdk
\end{mdbmargintb}%mdk%mdk


\end{document}
